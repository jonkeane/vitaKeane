%!TEX TS-program = lualatex
%!TEX encoding = UTF-8 Unicode

\documentclass[10pt, letterpaper]{article}
\usepackage[dvipsnames,usenames]{xcolor}
\usepackage{fontspec}

%%% Hanging indents: args length, number of lines.
%%% \begin{hangparas}{1em}{1}
\usepackage{hanging}
\usepackage{datetime}
\renewcommand{\dateseparator}{-}
\newdateformat{isodate}{%
\THEYEAR\dateseparator\twodigit{\THEMONTH}\dateseparator\twodigit{\THEDAY}}


\input{glyphtounicode}
\pdfgentounicode=1


% DOCUMENT LAYOUT
\usepackage{geometry}
\geometry{letterpaper, top=.65in, textwidth=6in, textheight=9.7in, marginparsep=1em}



% FONTS
\defaultfontfeatures{Mapping=tex-text} % converts LaTeX specials (``quotes'' --- dashes etc.) to unicode
\setmainfont [Ligatures={Common},Numbers=OldStyle, BoldFont={* Bold}, ItalicFont={* Italic}]{Minion Pro}
\setsansfont {Neutraface Text}
\setmonofont{Monaco}
% ---- CUSTOM AMPERSAND
\newcommand{\amper}{{\fontspec[Scale=.95]{Fontin}\&}}
\newcommand{\urltilde}{$\sim$}

% ---- MARGIN YEARS
\setlength{\marginparwidth}{0.5in}
\newcommand{\years}[1]{\marginpar{\scriptsize #1}}

% HEADINGS
\usepackage{sectsty}
\usepackage[normalem]{ulem}
\sectionfont{\sffamily\bfseries\upshape\Large}
\subsectionfont{\sffamily\bfseries\upshape\normalsize}
\subsubsectionfont{\sffamily\mdseries\upshape\normalsize}

% PDF SETUP
% ---- FILL IN HERE THE DOC TITLE AND AUTHOR
\usepackage[bookmarks, colorlinks, breaklinks]{hyperref}
\hypersetup{linkcolor=NavyBlue,citecolor=NavyBlue,filecolor=NavyBlue,urlcolor=NavyBlue,pdftitle=Jonathan Keane - CV,pdfauthor=Jonathan Keane,bookmarksopen=true}

% DOCUMENT
\begin{document}
%\pagestyle{empty}
\thispagestyle{empty} % No page number first page
\reversemarginpar
\raggedright

\begin{table}[!t]
  \begin{tabular*}{6.5in}{r|l}
    \hspace{3in}\textbf{\LARGE\sffamily Jonathan Keane}  &\\
Mattersight & \href{mailto:jonkeane@uchicago.edu}{\textsc{jonkeane@uchicago.edu}}\\
  200 \textsc{w} Madison Street
 & \href{http://jonkeane.com}{\textsc{jonkeane.com}}\\
 Suite 3100 & \\
Chicago, \textsc{il} 60606 & \\
  \end{tabular*}
\end{table}
%%% Begin with soft hyphen “\-” to line up the years marginpar.
\setlength\parindent{0in}
\setlength\parskip{0ex}
\pdfbookmark[0]{Current position}{Current position}
\section*{Current position}
\emph{Data Scientist}\\ \href{http://www.mattersight.com}{Mattersight}

\pdfbookmark[0]{Areas of specialization}{Areas of specialization}
\section*{Areas of specialization}
Data science, software development, predictive models, statistical computing, data visualization, articulatory phonetics and phonology, signed languages, American Sign Language \textsc{(asl)}, co-speech gesture, fingerspelling, syntax, morphology, variation, data analysis, computational approaches to linguistics

% \section*{Appointments held}
% \noindent
% \years{}

\pdfbookmark[0]{Education}{education}
\section*{Education}
\-\years{2014}PhD (with honors) in Linguistics, University of Chicago \\
\hspace{2em} \textit{Dissertation title} Towards an articulatory model of handshape:\\
\hspace{4em}What fingerspelling tells us about the phonetics and phonology of handshape in \textsc{asl}.\\
\hspace{2em} \textit{Advisors} Diane Brentari and Jason Riggle

%\-\years{2011}\textsc{MA} (anticipated) in Linguistics, University of Chicago

\-\years{2007}\textsc{BA} (Magna cum Laude) in Linguistics, University of Florida\\
\hspace{2em}\textit{\textsc{BA} Thesis} Nominal Incorporation in English \\
\hspace{2em}\textit{Advisor} D. Gary Miller

\pdfbookmark[0]{Honors + awards}{Honors + awards}
\section*{Awards + funding}
\-\years{2013--2015}\textit{\textsc{nsf} Doctoral Dissertation Research Improvement Grant} \\ Coarticulation and the phonetics of fingerspelling \\ \textsc{bcs}1251807

\-\years{2013}\textit{Nominated for best student presentation} at \textsc{tislr} 11 for \textit{Towards an articulatory model of handshape}

\-\years{2013--2014}\textit{Mellon Humanities Dissertation-Year Fellowship}

\-\years{2010}\textit{Rella I Cohn fund} Graduate student research funding -- \textsc{asl} fingerspelling data collection

\-\years{2008--}\textit{Graduate Aid Initiative} University of Chicago

\-\years{2007}\textit{University Scholars Program} University of Florida

\pdfbookmark[0]{Publications + talks}{Publications + talks}
\section*{Papers + talks}
\setlength\parskip{1ex}

<%= papers %>

%
% \subsection*{Software}
% \begin{hangparas}{1em}{1}
%
% \end{hangparas}

\pdfbookmark[0]{Experience}{Experience}
\section*{Experience}
\-\years{2014--2016}\textit{Post-Doctoral Fellow and Research Coordinator} \hfill University of Chicago \\ \href{http://gslcenter.uchicago.edu}{Center for Gesture, Sign, and Language} and  \href{http://linguistics.uchicago.edu}{Department of Linguistics}

\-\years{2011--2014}\textit{Lab Manager}  \href{http://signlanguagelab.uchicago.edu}{Sign Language Linguistics Laboratory} \hfill University of Chicago \\ Diane Brentari \\

\-\years{2009--2014}\textit{Research Assistant} \href{http://clml.uchicago.edu/}{Chicago Language Modeling Laboratory} \hfill University of Chicago \\Jason Riggle \\
Automated sign language recognition working group


\-\years{2006}\textit{Research Assistant} Variation in Control Structures \hfill University of Florida\\
Eric Potsdam

\pdfbookmark[0]{Teaching}{Teaching}
\section*{Teaching}
\-\years{2013  fall}\textit{Topic Instructor} Phonological Analysis 2 -- Jason Riggle \\
\textit{Topics} Data coding, data analysis, R, statistical methods

\-\years{2012 fall}\textit{Lecturer} Introduction to Linguistics \\

\-\years{2012 spring}\textit{Teaching Assistant} Introduction to Linguistics -- Ryan Bochnak\\
\textit{Guest lecture} Sign Language Linguistics

\-\years{2012 winter}\textit{Teaching Assistant} Introduction to Phonetics and Phonology -- Alan Yu\\

\-\years{2011 fall}\textit{Guest lecture} Sign Language Linguistics -- The phonetics of fingerspelling and its relation to \textsc{asl}

\-\years{2011 spring}\textit{Teaching Assistant} Introduction to Syntax -- Amy Dahlstrom\\
\textit{Guest lecture} Crosslinguistic syntactic reflexes of topic and focus


\pdfbookmark[0]{Service}{Service}
\section*{Service}
\-\years{2011--2013}\textit{Coordinator} Council on Advanced Studies workshop: Language, Cognition, and Computation

\-\years{2013}\textit{Reviewer} 49th Annual Meeting of the Chicago Linguistic Society

\-\years{2012}\textit{Reviewer} 48th Annual Meeting of the Chicago Linguistic Society

\-\years{2011}\textit{Graduate Coordinator} for prospective student recruitment activities

\-\years{2011}\textit{Reviewer} 47th Annual Meeting of the Chicago Linguistic Society

\-\years{2011}\textit{Graduate student liaison} to faculty

\-\years{2009--10}\textit{Officer} and \textit{conference organizer} Chicago Linguistic Society

\-\years{2010}\textit{Reviewer} 46th Annual Meeting of the Chicago Linguistic Society

\-\years{2009}\textit{Reviewer} 45th Annual Meeting of the Chicago Linguistic Society (Committee: Syntax)

\pdfbookmark[1]{Memberships}{Memberships}
\subsection*{Memberships}
\label{sec:memberships}
\-\years{2013--}Sign Language Linguistics Society

\-\years{2008--}Linguistic Society of America

\pdfbookmark[0]{Languages}{Languages}
\section*{Languages}
\label{sec:languages}
English~(native), American Sign Language~(fluent), German~(reading), Swedish~(basic), Japanese~(basic), Hebrew~(basic)

\pdfbookmark[0]{Computer languages + applications}{Computer languages + applications}
\section*{Computer languages + applications}
\label{sec:computer}
Python, R, {\sc sql}, \LaTeX, {\sc (x)html}, \textsc{php}, C, \textsc{elan}, microcontrollers, emacs, OpenOffice, Adobe Photoshop, Adobe Illustrator, Adobe InDesign, Mac OS X, Linux/\textsc{unix}, PhaseSpace motion capture equipment

%\vspace{1cm}
\vfill{}
\hrulefill
\begin{center}
{\footnotesize \href{http://cv.jonkeane.com}{cv.jonkeane.com} — Last updated: \isodate\today}
\end{center}

\end{document}



%%%%%%% TEMPLATE FROM:
%------------------------------------
% Dario Taraborelli
% Typesetting your academic CV in LaTeX
%
% URL: http://nitens.org/taraborelli/cvtex
% DISCLAIMER: This template is provided for free and without any guarantee
% that it will correctly compile on your system if you have a non-standard
% configuration.
%------------------------------------
